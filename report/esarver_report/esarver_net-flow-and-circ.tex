\documentclass[conference]{IEEEtran}
\IEEEoverridecommandlockouts
% The preceding line is only needed to identify funding in the first footnote. If that is unneeded, please comment it out.
\usepackage{cite}
\usepackage{amsmath,amssymb,amsfonts}
\usepackage{algorithm}
\usepackage{algorithmicx}
\usepackage{algpseudocode}
\usepackage{tikz}
\usetikzlibrary{arrows.meta, positioning}
\usepackage{listings}
\usepackage{graphicx}
\usepackage{textcomp}
\usepackage{xcolor}
\def\BibTeX{{\rm B\kern-.05em{\sc i\kern-.025em b}\kern-.08em
    T\kern-.1667em\lower.7ex\hbox{E}\kern-.125emX}}
\begin{document}

\title{Network Flow and Circulation with Demands}


\author{\IEEEauthorblockN{Edwin Sarver}}

\maketitle

\section{Introduction}
% TODO 

\section{Algorithms}\label{algo}
Two main algorithms were used in this project: Breadth-First Search (BFS),
and Edmonds-Karp, a derivative work of Ford-Fulkerson.

\subsection{Breadth-First Search}
The breadth-first search algorithm is used to find the path from the given
start node to the given end node that includes the fewest edges. This is 
accomplished by visiting all adjacent nodes to the source, then all the 
adjacent nodes to those nodes and so on until all nodes have been visited.

When performing these steps, it is necessary to note which nodes have been 
visited, the number of jumps to get to that node, and the order in which 
the nodes were visited. A pseudocode version of BFS is shown in 
Algorithm \ref{bfs} which shows how BFS works. 

\begin{algorithm}
\caption{Breadth-First Search \cite{b1}}\label{bfs}
   \begin{algorithmic}[1]
    \Function{BFS}{$G, s$}
    \For{node in $G.V$} \label{bfs_init_loop}
        \State $node.color = $WHITE
        \State $node.dist = \infty$
        \State $node.prev =$ NIL
    \EndFor
    \State $s.color =$ GRAY
    \State $s.dist =$ 0
    \State $s.prev =$ NIL
    \State $Q = \emptyset$
    \State Enqueue($Q, s$)
    \While{$Q \neq \emptyset$} \label{bfs_while_loop}
    	\State $u =$ Dequeue($Q$)
    	\For{each $v \in u.adjacent\_nodes$} \label{bfs_nested_for}
    		\If{$v.color ==$ WHITE}
    			\State $v.color =$ GRAY
    			\State $v.dist = u.dist + 1$
    			\State $v.prev = u$
    			\State Enqueue($Q, v$)
    		\EndIf
    	\EndFor
		\State $u.color =$ BLACK
    \EndWhile
    \EndFunction
   \end{algorithmic}
\end{algorithm} 

\subsubsection{Theoretical Performance}
The initialization loop (line \ref{bfs_init_loop}) of BFS in Algorithm \ref{bfs} 
will complete in $O(V)$ time because it must run over every vertex in $V$. 
The while loop (line \ref{bfs_while_loop}) will run over each edge in the 
graph and will thus complete in $O(E)$. Therefore the total running time is
$O(V+E)$.


\subsection{Ford-Fulkerson/Edmonds-Karp}
The Ford-Fulkerson method is a set of algorithms used to determine the maximum 
possible flow through a network. There is a derivative method called Edmonds-Karp 
that prescribes that a path-finding algorithm be used to find an augmenting path. 
The method works as shown in Algorithm \ref{ff}.
 

\begin{algorithm}
\caption{Ford-Fulkerson Method \cite{b1}}\label{ff}
	\begin{algorithmic}[1]
		\Function{FordFulkerson}{$G, s, t$}
			\State $f = 0$
			\While{$\exists p \in G_f$}
				\State $f = G_f.augment(p)$
			\EndWhile
			\State \Return $f$
		\EndFunction
	\end{algorithmic}
\end{algorithm}

The first part in Algorithm \ref{ff} is to find an augmenting path. An augmenting 
path is a simple path, one that only visits the consituent nodes and edges once, 
from the source node to the sink node. The flow along an augmenting path can only be 
increased by the minimum capacity along that path, along what is known as the critical
edge. 

After finding the first augmenting path and increasing the flow along that path, a 
residual network is created. The residual graph is created by removing the critical
all critical edges, and adding in edges with residual capacities defined by Equation 
\ref{residual_capacity}.

\begin{equation}
\label{residual_capacity}
c_f = \begin{cases}
		c(u,v) - f(u,v) & if (u,v) \in E,\\
		f(v,u) & if (v,u) \in E,\\
		0 & otherwise.
	\end{cases}
\end{equation}

\subsubsection{Theoretical Performance} % TODO revisit this.
When the flow on an edge in an augmenting path is equal to the capacity of 
that edge, that edge is said to be "critical". A critical edge, after augmenting
occurs, is removed from the residual graph. A given critical edge cannot become
critical again until the shortest path from the source to the sink is 2 greater 
than when that critical edge was found. Any edge in a graph can be critical
at most $V/2$ times and there are $E$ edges. Therefore the time complexity of 
the Edmonds-Karp algorithm is $O(EV)$. The loop in the Ford-Fulkerson can be run
up to $O(E)$ times, and therefore the total time complexity is $O(VE^2)$.

\subsection{Circulation}
The Circulation of supply and demand problem is a logistical problem in 
which a flow network has various nodes that have a given demand value. 
If that demand value is negative, that node has a supply. If the demand
value is positive, that node has a demand.

The problem can be reformulated to be a maximum flow problem. Each negative 
demand value can be thought of as an edge from a source to the node with the
demand value. That edge will have a capacity equal to the absolute value of 
the demand value. Similarly, each positive demand value can be thought of as
an edge from the node with the demand value to a sink node with a capacity 
equal to the demand value. 

If the total of all the supply values is not equal to the demand value, there 
is no circulation in the system. However, if the supply and demand values are 
equivalent, then the maximum flow of the system must be determined. The Edmonds-Karp
algorithm can be used to get the maximum flow. 

If the maximum flow is not equal to the sum of the supply or demand values, there is no
circulation in the system. Only if the maximum flow is equal to the sum of the supply value 
and the sum of the demand values is circulation present in the system. 

\section{Data Structures}
Graphs are represented by structures that show how the graphs are connected. There are 
multiple ways to represent these connections. One way is to use an adjacency matrix.

In an adjacency matrix, the connection between each node to every other node is represented.
In such a structure, there has to be $|V|^2$ slots to hold the connection information. This
method of representation is very convenient for the programmer because determining if a 
connection exists is easy. However, it is only considered space efficient if the graph in
question is very dense. 

The other option is to use an adjacency list. Each node in an adjacency list contains a list 
that holds onle the connection information for the nodes to which is is connected. This makes 
adjacency lists well suited for sparse graphs. 

Since flow networks are sparse graphs, adjacency lists are the obvious choice to represent them.

\section{Implementation Difficulties} % TODO revisit this
There were very few obstacles while implementing the algorithms described in section 
\ref{algo}. One issue that was narrowly avoided, however, was ensuring that the circulaltion
graph implementation was able to correctly inherit from the other graph objects. There
were several methods that had to be \lstinline{virtual}. 

This would have cause an issue in the Ford-Fulkerson implementation if a circulation graph
was passed to it, such as when the circulation algorithm was run. This could have been 
worked around my converting the circulation graph to a flow network, but would have relied
too heavily on convention to ensure errors were not made.


\section{Experimental Performance Analysis}
% TODO

\section{Insights}
% TODO

\section{Test Cases}
% TODO
\subsection{Design}
% TODO
\subsection{Results}
% TODO

\section{Conclusion}

\begin{thebibliography}{00}
    \bibitem{b1} T. H. Cormen, C. E. Leiserson, R. L. Rivest, and C. Stein, Introduction to Algorithms, 3rd ed., The MIT Press, 2009, pp. 594--602, 709--731 
\end{thebibliography}

\end{document}
