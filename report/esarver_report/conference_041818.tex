\documentclass[conference]{IEEEtran}
\IEEEoverridecommandlockouts
% The preceding line is only needed to identify funding in the first footnote. If that is unneeded, please comment it out.
\usepackage{cite}
\usepackage{amsmath,amssymb,amsfonts}
\usepackage{algorithm}
\usepackage{algorithmicx}
\usepackage{algpseudocode}
\usepackage{tikz}
\usetikzlibrary{arrows.meta, positioning}
\usepackage{graphicx}
\usepackage{textcomp}
\usepackage{xcolor}
\def\BibTeX{{\rm B\kern-.05em{\sc i\kern-.025em b}\kern-.08em
    T\kern-.1667em\lower.7ex\hbox{E}\kern-.125emX}}
\begin{document}

\title{Network Flow and Circulation with Demands}


\author{\IEEEauthorblockN{Edwin Sarver}}

\maketitle

\section{Introduction}
% TODO 

\section{Algorithms}
Two main algorithms were used in this project: Breadth-First Search (BFS),
and Edmonds-Karp, a dirivative work of Ford-Fulkerson.

\subsection{Breadth-First Search}
The breadth-first search algorithm is used to find the path from the given
start node to the given end node that includes the fewest edges. This is 
accomplished by visiting all adjacent nodes to the source, then all the 
adjacent nodes to those nodes and so on until the destination node is found.

When performing these steps, it is necessary to note which nodes have been 
visited, the number of jumps to get to that node, and the order in which 
the nodes were visited. Algorithm \ref{bfs} accomplishes that task. 

\begin{algorithm}
\caption{Breadth-First Search \cite{b1}}\label{bfs}
   \begin{algorithmic}
    \Function{BFS}{$G, s$}
    \For{node in $G.V$}
        \State $node.color = $WHITE
        \State $node.dist = \infty$
        \State $node.prev =$ NIL
    \EndFor
    \State $s.color =$ GRAY
    \State $s.dist =$ 0
    \State $s.prev =$ NIL
    \State $Q = \emptyset$
    \State Enqueue($Q, s$)
    \While{$Q \neq \emptyset$}
    	\State $u =$ Dequeue($Q$)
    	\For{each $v \in u.adjacent\_nodes$}
    		\If{$v.color ==$ WHITE}
    			\State $v.color =$ GRAY
    			\State $v.dist = u.dist + 1$
    			\State $v.prev = u$
    			\State Enqueue($Q, v$)
    		\EndIf
    	\EndFor
		\State $u.color =$ BLACK
    \EndWhile
    \EndFunction
   \end{algorithmic}
\end{algorithm} 

\subsection{Ford-Fulkerson/Edmonds-Karp}
The Ford-Fulkerson method is a set of algorithms used to determine the maximum 
possible flow through a network. There is a derivative method called Edmonds-Karp 
that prescribes that a path-finding algorithm be used to find an augmenting path. The method works as shown in Algorithm \ref{ff}.
 

\begin{algorithm}
\caption{Ford-Fulkerson Method \cite{b1}}\label{ff}
	\begin{algorithmic}
		\Function{FordFulkerson}{$G, s, t$}
			\State $f = 0$
			\While{$\exists p \in G_f$}
				\State $f = G_f.augment(p)$
			\EndWhile
			\State \Return $f$
		\EndFunction
	\end{algorithmic}
\end{algorithm}

The first part in Algorithm \ref{ff} is to find an augmenting path. An augmenting 
path is a simple path, one that only visits the consituent nodes and edges once, 
from the source node to the sink node. The flow along an augmenting path can only be 
increased by the minimum capacity along that path. 

After finding the first augmenting path and increasing the flow along that path, a 
residual network is created. A residual network is created by the residual capacity defined by Equation \ref{residual_capacity}.

\begin{equation}
\label{residual_capacity}
c_f = \begin{cases}
		c(u,v) - f(u,v) & if (u,v) \in E,\\
		f(v,u) & if (v,u) \in E,\\
		0 & otherwise.
	\end{cases}
\end{equation}

\subsection{Circulation}
% TODO

\section{Data Structures}
% TODO

\section{Implementation Difficulties}
% TODO

\section{Performance Analysis}
% TODO
\subsection{Theoretical}
% TODO
\subsection{Experimental}
% TODO

\section{Insights}
% TODO

\section{Test Cases}
% TODO
\subsection{Design}
% TODO
\subsection{Results}
% TODO

\section{Conclusion}

\begin{thebibliography}{00}
    \bibitem{b1} T. H. Cormen, C. E. Leiserson, R. L. Rivest, and C. Stein, Introduction to Algorithms, 3rd ed., MIT, 2009, pp. 594--602, 709--731 
\end{thebibliography}

\end{document}
